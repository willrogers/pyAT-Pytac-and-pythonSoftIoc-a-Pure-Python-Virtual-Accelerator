\documentclass[a4paper,
               %boxit,        % check whether paper is inside correct margins
               %titlepage,    % separate title page
               %refpage       % separate references
               %biblatex,     % biblatex is used
               keeplastbox,   % flushend option: not to un-indent last line in References
               %nospread,     % flushend option: do not fill with whitespace to balance columns
               %hyphens,      % allow \url to hyphenate at "-" (hyphens)
               %xetex,        % use XeLaTeX to process the file
               %luatex,       % use LuaLaTeX to process the file
               ]{jacow}
%
% ONLY FOR \footnote in table/tabular
%
\usepackage{pdfpages,multirow,ragged2e} %
\usepackage{listings}
\lstset{basicstyle=\footnotesize\ttfamily,breaklines=true}
\lstset{framextopmargin=50pt}
\lstset{columns=fullflexible}
%
% CHANGE SEQUENCE OF GRAPHICS EXTENSION TO BE EMBEDDED
% ----------------------------------------------------
% test for XeTeX where the sequence is by default eps-> pdf, jpg, png, pdf, ...
%    and the JACoW template provides JACpic2v3.eps and JACpic2v3.jpg which
%    might generates errors, therefore PNG and JPG first
%
\makeatletter%
	\ifboolexpr{bool{xetex}}
	 {\renewcommand{\Gin@extensions}{.pdf,%
	                    .png,.jpg,.bmp,.pict,.tif,.psd,.mac,.sga,.tga,.gif,%
	                    .eps,.ps,%
	                    }}{}
\makeatother

% CHECK FOR XeTeX/LuaTeX BEFORE DEFINING AN INPUT ENCODING
% --------------------------------------------------------
%   utf8  is default for XeTeX/LuaTeX
%   utf8  in LaTeX only realises a small portion of codes
%
\ifboolexpr{bool{xetex} or bool{luatex}} % test for XeTeX/LuaTeX
 {}                                      % input encoding is utf8 by default
 {\usepackage[utf8]{inputenc}}           % switch to utf8

\usepackage[USenglish]{babel}

%
% if BibLaTeX is used
%
\ifboolexpr{bool{jacowbiblatex}}%
 {%
  \addbibresource{jacow-test.bib}
  \addbibresource{biblatex-examples.bib}
 }{}
\listfiles

\begin{document}

\title{pyAT, Pytac and pythonSoftIoc: a Pure Python Virtual Accelerator} 

\author{W. Rogers\thanks{will.rogers@diamond.ac.uk}, T. J. R. Nicholls, A. A. Wilson, Diamond Light Source, Oxfordshire, UK}
	
\maketitle

\begin{abstract}
Virtual accelerators are used for testing control system software against
realistic accelerator simulations. Previous virtual accelerators for 
synchrotron light sources have used Tracy \cite{tracy} and Elegant \cite{elegant} as the simulator,
but without Python bindings for accelerator simulations it has been difficult 
to create a virtual accelerator using Python. With the development of 
Python Accelerator Toolbox (pyAT) \cite{pyat}, that is now possible. This paper 
describes the combination of pyAT, Python Toolkit for Accelerator Controls 
(Pytac) and pythonSoftIoc to create an EPICS-based virtual accelerator for 
Diamond Light Source.
\end{abstract}

\section{Motivation}

High-level control system software is designed to interact with the control system
that is connected to the real hardware. Testing this software can be inconvenient
for two main reasons: during design and commissioning the hardware might not
yet exist, and during its operational lifetime the hardware is in use most of
the time.

A \textit{virtual accelerator} is an application that allows testing a control
system by providing the same interface as a subset of the control system that 
is required for the operation of high-level applications. Although it is possible
to provide dummy values for different control system parameters, it is much more 
useful to combine those parameters with a simulation so that they respond in a 
physically accurate way to any changes. The software under test then requires no 
changes in order to run against a virtual accelerator.

Python has a number of advantages for developing a virtual accelerator:
it is free and open source, is very widely used in both science and industry,
has many useful third-party libraries available, is simple to start using
and is capable of building large applications that scale well.

To build this virtual accelerator we needed a number of components:

\begin{itemize}
    \item a simulation code for synchrotron light sources that can be called from Python
    \item a Python framework that understands the elements of a particle accelerator 
    \item the ability to convert between engineering units (used in the control system) and physics
  units (used in simulation codes)
    \item a server that can hook the Python code into the control system
\end{itemize}

The following sections describe these components.

\section{pyAT}

Accelerator Toolbox (AT) \cite{at} is a simulation code for synchrotron light sources
developed for use in Matlab. Its numerical engine is based around
'integrators' that calculate the effect of a particle in 6D phase space.
For efficiency purposes these integrators were written in C and compiled for
use in Matlab.

This design allowed for the same integrators to be compiled for use in Python
code. pyAT uses the same numerical engine as AT, with Python classes and functions
written to derive accelerator parameters from the engine. It uses the libraries
NumPy and SciPy for several numerical utilities.

pyAT provides a number of features. Lattice and element types are defined and may
be loaded and saved to different file formats. It performs particle tracking and 
allows calculation of transfer matrices and closed orbit with radiation included
or excluded. Other derived parameters that are calculated include linear optics,
radiation integrals and detail of the beam envelope. It includes a plotting package
that uses Matplotlib to provide various plotting functions.

Testing has shown pyAT to give exactly the same numerical results as AT with 
a speed comparable to other accelerator simulations \cite{pyat}.

\section{Pytac}

Python Toolkit for Accelerator Controls (Pytac) is a Python library designed to
enable working with the different parts of accelerators. Each element in an
accelerator is represented by an object that may have one or more 'fields' 
corresponding to physical parameters. The ability to address the different 
elements of an accelerator by position in the accelerator and element family
is often useful in high-level applications and Python scripts.

\begin{lstlisting}
>>> bpm1 = lattice.get_elements('BPM')[0]
>>> bpm1.fields()[pytac.LIVE]
['enabled', 'x', 'y']
>>> bpm1.get_value('x', data_source=pytac.LIVE)
-2.1e-5
\end{lstlisting}

 Pytac allows requesting the live values of the parameters from the accelerator
control system. It also allows efficiently requesting the values for an entire
family.

\begin{lstlisting}
>>> lattice.get_element_values('BPM', 'x')
[-4.6e-05, 8.2e-05, 7e-05, ...]
\end{lstlisting}

Many of the ideas in Pytac were inspired by a similar application Matlab Middle
Layer (MML) \cite{mml}.

\subsection{Unit Conversion}

It is often useful to use the same accelerator parameter in different unit systems:
the units used by physicists for calculations, and the units used by the control
system for control and monitoring. Pytac has a built-in unit conversion mechanism
that allows requesting and setting parameters in either of these unit systems. The
following example requests a quadrupole setting in engineering units ($A$) and
physics units ($Tm^{-1}$):

\begin{lstlisting}
>>> quad.get_value('b1', units=pytac.ENG)
103.18108367919922
>>> quad.get_value('b1', units=pytac.PHYS)
-1.0192934647760261
\end{lstlisting}

Two unit conversion mechanisms are currently available: polynomial (often used for 
linear conversions) and piecewise cubic hermite interpolating polynomial (PCHIP),
an algorithm that allows smoothed interpolation between arbitrary measured data
points. Implementing further unit conversion mechanisms is simple.

\subsection{Configuration}

The accelerator definition is stored in a number of CSV files. These are easy
to edit and are efficiently stored in the Git version control system. The
configurations for the Diamond accelerators are exported using a Matlab script
from the existing configurations that are set up in MML.

\section{ATIP}

Accelerator Toolbox Interface for Pytac (ATIP) is the adapter that makes pyAT
available as a simulator for Pytac. Once loaded, pyAT provides an online model
for interactive use in Pytac.

\begin{lstlisting}
>>> lattice.set_default_data_source(pytac.SIM)
>>> lattice.get_element_values('BPM', 'x')
[0.0, 0.0, 0.0, ...]
>>> h_corr = lattice.get_elements('HSTR')[0]
>>> h_corr.set_value('x_kick', 0.1, units=pytac.ENG)
>>> lattice.get_element_values('BPM', 'x')
[0.24630504031808942, 0.12495575893699563, 
 -0.1257213016476168, ...]
\end{lstlisting}

ATIP has a simple threading mechanism. Any changes that would require a recalculation
using pyAT are placed on a queue. A simulation thread loops checking whether there
are any items in the queue. If so, it empties the queue, applies each change to its
model and recalculates. When a request for data is received, ATIP will check whether
an update is pending and if so will wait until the recalculation is complete before
returning that data.

\begin{figure}[!hbt]
    \centering
    \includegraphics*[width=\columnwidth]{MOPHA017f1}

    \caption{How ATIP integrates into Pytac}
    \label{fig:atip}
\end{figure}

\section{pythonSoftIoc}

At Diamond Light Source we use the EPICS distributed control system, with many
servers known as IOCs that provide some subset of the control system parameters
(process variables, or PVs). Many IOCs are embedded devices that interact
directly with hardware, but often it is useful to make standalone IOCs that may
not represent hardware devices at all; these are called \textit{soft IOCs}.
pythonSoftIoc \cite{pythonioc} is a Python library that allows creating an EPICS IOC
using only Python code. The virtual accelerator uses this library to create
an IOC using the PV names used on the Diamond accelerator, then respond
appropriately to any interactions.

\section{A Pure Python Virtual Accelerator}

Using the tools above, it is now possible to assemble a virtual accelerator
using only Python. The definition of the accelerator is provided by Pytac.
The accelerator simulation is provided by pyAT. The EPICS IOC is provided by 
pythonSoftIoc.

This virtual accelerator is now being used to test the following high-level applications
at Diamond.

\subsection{High-Level Applications}

\subsubsection{Slow Orbit Feedback}

The slow orbit feedback system is well simulated using the virtual accelerator.
The beam position is simulated and returned via the BPM elements; the feedback
system calculates a correction to be applied to the corrector magnets over EPICS
and the simulation is updated once the correction has been applied. The \SI{1}{Hz} rate
of the slow feedback system can be handled by the virtual accelerator.

\subsubsection{RF Feedback} 

The orbit feedback systems correct the beam orbit so that the BPM readings are zero.
However, there are multiple solutions to this, some of which may have forced the
electrons to a different energy than designed. The RF feedback system determines
whether the net effect of the corrector magnets includes such a change; if so, it
changes the RF frequency to remove it. The slow orbit feedback and RF feedback 
systems work together, and can both be tested against the virtual accelerator.

\subsubsection{Tune Feedback}

Diamond's tune feedback system \cite{feedbacks} uses a subset of the quadrupoles 
to correct variations in the tunes. A response matrix dictates how the tunes 
change when the quadrupole settings change; this matrix is inverted to determine 
the quadrupole changes required to correct a deviation in the vertical and 
horizontal tunes.

\subsubsection{Vertical Emittance Feedback}

Diamond runs a vertical emittance feedback system \cite{feedbacks} that keeps the beam 
size broadly constant. Since pyAT provides the emittance value for the ring using the 
Ohmi Envelope formalism, it is possible to use the virtual accelerator to test the 
vertical emittance feedback system.

\subsubsection{BURT}

The Back Up and Restore Tool (BURT) is used at Diamond to save and restore machine
configuration in the form of stored values for specific PVs. We have used the virtual
accelerator to test BURT functionality as we develop a new version of Burt in Python.
Burt may also be used to save different configurations of the virtual accelerator.

\subsection{Challenges}

There are a number of details of the high level applications that make using the
virtual accelerator challenging.

\subsubsection{Speed of calculation and update rates}

A fundamental limitation of this design of virtual accelerator is the elapsed time
taken in order to recalculate machine parameters. These simulations are typically
run on desktop machines.

The simulation used for the virtual accelerator has the two parts summarised in 
Table~\ref{simfunctions}. This gives an approximate update rate of \SI{1}{Hz}, sufficient
for most of the high-level applications.

The one example for which this caused problems was the vertical emittance feedback
system. The PV that provides the vertical emittance value updates at about \SI{5}{Hz},
and the feedback system reports slower updates as a failure. As Python is inherently
single-threaded it is impossible to update a PV on the virtual accelerator while 
the recalculation is taking place; in any case, unless there are valid new values to
report more frequent updates may cause feedback systems to misbehave.

The vertical emittance feedback system can be tested against the virtual accelerator
if the update check is disabled; whether this check is useful in the application itself is
being considered.

\begin{table}[!hbt]
   \centering
   \caption{Virtual Accelerator simulation functions executed on a typical desktop PC}
   \begin{tabular}{lcc}
       \toprule
       \textbf{Function} & \textbf{Description}                      & \textbf{Runtime} \\
       \midrule
           \texttt{linopt()}         & Derive linear optics            & \SI{0.25}{s}        \\
           \texttt{ohmi\_envelope()}  & Calculate emittance      & \SI{0.65}{s}        \\
       \bottomrule
   \end{tabular}
   \label{simfunctions}
\end{table}

\subsubsection{Control system complexities}

Sometimes the EPICS Control System is more complicated than the simple view presented
by a virtual accelerator.

One example at Diamond is the way that the quadrupoles are 
controlled, which uses a number of PVs to aggregate contributions to the magnet
setpoint from different sources, one of which is the tune feedback system \cite{feedbacks}.
In this case it was possible to test the tune feedback system by providing 'mirror'
PVs that respond in the same way to the original setpoint PVs; in other cases
simple transforms can be applied to the PV value.

Certain PVs at Diamond are available as individual PVs per element but are also
available for convenience as a waveform PV: for example, there are two waveform
PVs containing all 173 BPM values in the horizontal and vertical planes. The natural
way for the virtual accelerator to provide this information is per-element, but
applications are more likely to use the waveforms. To solve this we added a
configuration mechanism to aggregate individual values into addtional waveform PVs.

These transformations are handled by classes in ATIP, and it would be possible to
make other transformations by writing similar classes.

\section{Notes on software}

All of the components described above are open source and the source code is available
on Github. pyAT, Pytac and ATIP are available on the Python Package Index (PyPI) \cite{pypi} 
for simple installation using pip. Any interest in using or collaborating on these 
projects would be welcomed.

Jupyter notebooks are a good way of demonstrating the capabilities of these applications.
Some example notebooks exist in the Git repositories, and further examples are being developed.

Python versions 2.7 and 3.5+ are supported, although Python 2 support will be deprecated
in the near future in line with the approach of the rest of the Python community.

\section{Further work}

The components described above begin to form a versatile toolkit for accelerator
physics and control system applications using Python. 

The next application in this toolkit is Visualiser and Optimiser for Linear Optics
(Volo), which uses ATIP and pyAT to allow interactive lattice viewing and editing
via a PyQt GUI. Most of the capability for this tool exists in the components
described above, meaning that the Volo project mostly requires using these components
and constructing an intuitive user interace. A prototype of this application is under
development.

\begin{figure}[!hbt]
    \centering
    \includegraphics*[width=\columnwidth]{MOPHA017f2}
    \caption{A screenshot from an early version of Volo}
    \label{fig:volo}
\end{figure}

\begin{thebibliography}{9}                                                      
                  
\bibitem{tracy}
    Nishimura, Hiroshi. (1988). TRACY: A tool for accelerator design and analysis (LBL--25236)

\bibitem{elegant}                                                               
    M. Borland, "elegant: A Flexible SDDS-Compliant Code for Accelerator Simulation," Advanced Photon Source LS-287, September 2000.
      
\bibitem{pyat}
    W. Rogers, N. Carmignani, L. Farvacque, and B. Nash,
    \textquotedblleft{pyAT: A Python Build of Accelerator Toolbox}\textquotedblright,
    in \emph{Proc. 8th Int. Particle Accelerator Conf. (IPAC'17)}, 
    Copenhagen, Denmark, May 2017

\bibitem{mml}
     Portmann, G. et al, "An Accelerator Control Middle Layer Using Matlab", 
     in Conf.Proc. C0505161 (2005) 4009 SLAC-PUB-11445, LBNL-PUB-925, PAC-2005-FPAT077 
     
\bibitem{at}                                                                    
    A. Terebilo, "Accelerator Toolbox for MATLAB", SLAC-PUB-8732, 2001.   
    
\bibitem{feedbacks}
  M.~Heron {\it et al.},
  "Feed-forward and Feedback Schemes applied to the Diamond Light Source Storage Ring",
      in \emph{Proc. 8th Int. Particle Accelerator Conf. (IPAC'14)}, 
    Dresden, Germany, May 2014

\bibitem{pythonioc}                                                                  
    pythonSoftIOC on Github, \url{https://github.com/Araneidae/pythonIoc} 
    
\bibitem{pypi}                                                                  
    PyPI, \url{https://pypi.python.org/pypi} 
                                                  
\end{thebibliography}                                                 

\end{document}
